\documentclass[11pt,a4paper]{article}
\usepackage{hyperref}
\usepackage[left=1.5in,right=1.5in,top=1.5in,bottom=1.5in]{geometry}
\usepackage{enumitem}
\usepackage{amssymb,amsmath,mathrsfs,amsthm, empheq}
\usepackage{parskip, graphicx, float, wrapfig, subcaption}

\setlength{\parindent}{0pt}
\setlength{\parskip}{3mm}

\pagestyle{empty}

%Title and formatting stuff
\usepackage{titlesec}
\titleformat{\chapter}
{\normalfont\Huge\color{aaltoBlue}}{\thechapter}{20pt}{\filleft\Huge}[\vskip 0.5ex\titlerule]
\usepackage{fancyhdr}
\pagestyle{fancy}
\fancyhead[L]{}
\newcommand{\mymarginpar}[1]{\marginpar{\it \small \raggedleft #1}}
\renewcommand*\thesection{\arabic{section}}

% Miscellaneous LaTeX macros
\newcommand{\heading}[1]{{\Large\color{blue!80!black}\textbf{#1}}\par}
\newcommand{\slemp}[1]{{\color{red}#1}}
\newcommand{\slred}[1]{{\color{red}#1}}
\newcommand{\slbf}[1]{{\color{blue}\textbf{#1}}}
\newcommand{\pair}[2]{\langle {#1}, {#2} \rangle}
\newcommand{\Abar}{\bar{A}}
\newcommand{\ALG}{\text{ALG}}
\newcommand{\approxA}{\alpha_{\cal A}}
\newcommand{\argmin}{\operatornamewithlimits{arg\ min}}
\newcommand{\assign}{\mbox{$\; \leftarrow \;$}}
\newcommand{\bbar}{\bar{b}}
\newcommand{\Bin}{\text{Bin}}
\newcommand{\calA}{{\cal A}}
\newcommand{\calC}{{\cal C}}
\newcommand{\calI}{{\cal I}}
\newcommand{\calM}{{\cal M}}
\newcommand{\calN}{{\cal N}}
\newcommand{\calS}{{\cal S}}
\newcommand{\chat}{\hat{c}}
\newcommand{\chatopt}{\hat{c}^*}
\newcommand{\copt}{c^*}
\newcommand{\Cov}{\text{Cov}}
\newcommand{\DC}{\text{DC}}
\newcommand{\delC}{\partial C}
\newcommand{\delS}{\partial S}
\newcommand{\delT}{\partial T}
\newcommand{\diag}{\textit{diag}}
\newcommand{\dtv}{d_{\text{V}}}
\newcommand{\dom}{\mathcal{D}}
\newcommand{\E}{\text{E}}
\newcommand{\eps}{\varepsilon}
\newcommand{\false}{\text{false}}
\newcommand{\Gbar}{\bar{G}}
\newcommand{\Gnp}{{\cal G}(n,p)}
\newcommand{\Int}{\mathbf{I}}
\newcommand{\lambdamax}{\lambda_{\text{max}}}
\newcommand{\LPI}{$\text{LP}_{\text{I}}$}
\newcommand{\LPII}{$\text{LP}_{\text{II}}$}
\newcommand{\NN}{\mathbb{N}}
\newcommand{\om}{\omega}
\newcommand{\Om}{\Omega}
\newcommand{\OPT}{\text{OPT}}
\newcommand{\OPTf}{\text{OPT}_f}
\newcommand{\phase}{\bar{\sigma}}
\newcommand{\phat}{\hat{p}}
\newcommand{\pt}{p^{(t)}}
\newcommand{\rbot}{r^{\bot}}
\newcommand{\rhat}{\hat{r}}
\newcommand{\RENT}{\text{RENT}}
\newcommand{\RR}{\mathbb{R}}
\newcommand{\RRmn}{\mathbb{R}^{m\times n}}
\newcommand{\RRn}{\mathbb{R}^n}
\newcommand{\RRnn}{\mathbb{R}^{n\times n}}
\newcommand{\sgn}{\text{sgn}}
\newcommand{\stot}{$s$-$t$}
\newcommand{\true}{\text{true}}
\newcommand{\st}{\text{s.t.}}
\newcommand{\vivi}{v_i^Tv_i}
\newcommand{\vivj}{v_i^Tv_j}
\newcommand{\Var}{\text{Var}}
\newcommand{\xbar}{\bar{x}}
\newcommand{\xopt}{x^*}
\newcommand{\ybar}{\bar{y}}
\newcommand{\yopt}{y^*}
\newcommand{\zbar}{\bar{z}}
\newcommand{\zopt}{z^*}
\newcommand{\ZIP}{Z^*_{\text{IP}}}
\newcommand{\ZLP}{Z^*_{\text{LP}}}
\newcommand{\ZZ}{\mathbb{Z}}

\newcommand{\tD}[1]{\textrm{\em\color{aaltoGreen}{#1}}}
\newcommand{\tB}[1]{\text{\color{aaltoBlack}{#1}}}
\newcommand{\Pt}{\ensuremath{\mathrm{P}}}
\newcommand{\NP}{\ensuremath{\mathrm{NP}}}
\newcommand{\bra}{\ensuremath{\langle}}
\newcommand{\ket}{\ensuremath{\rangle}}
\newcommand{\DFT}{\ensuremath{\operatorname{DFT}}}
\newcommand{\rev}{\ensuremath{\operatorname{rev}}}
\newcommand{\quo}{\ensuremath{\operatorname{quo}}}
\newcommand{\rem}{\ensuremath{\operatorname{rem}}}
\newcommand{\take}{\ensuremath{\operatorname{\upharpoonright}}}
\newcommand{\ord}{\ensuremath{\operatorname{ord}}}
\newtheorem{fact}{Fact}
\newcommand\Edit{\mathrm{Edit}}
\newcommand\backtrack{\stackrel{\text{back-track}}{\longrightarrow}}

\newcommand\N{\mathbb{N}} % Sorry did not see \NN etc earlier and now my \N  is everywhere :(
\newcommand\R{\mathbb{R}}
\newcommand\Z{\mathbb{Z}}

%--------- definition environment commands --

\theoremstyle{definition}
\newtheorem{definition}{Definition}[section]
\newtheorem{lemma}{Lemma}[section]
\newtheorem{corollary}{Corollary}[section]
\newtheorem{theorem}{Theorem}[section]
\newtheorem{example}{Example}[section]

%--------- STUFF FOR 2021 ITERATION ---------

\usepackage{bbm}
\usepackage{float}
\usepackage{tikz}
\newcommand{\ex}{\text{E}}
\newcommand{\pr}{\text{P}}
\allowdisplaybreaks

% question title
\newcommand{\qtitle}[1]{{\color{aaltoBlue}\textbf{#1}}}

%% algorithm environment
\usepackage[ruled, vlined]{algorithm2e}
\newcommand\mycommfont[1]{\footnotesize\ttfamily\textcolor{aaltoBlue}{#1}}
\SetCommentSty{mycommfont}
\SetFuncSty{text}
\SetKwProg{Fn}{def}{:}{}
\usepackage{ulem}

\newif\ifanswer
\newif\ifgraded
\newcommand{\InsertAnswer}[1]{\vspace{1mm}{\color{aaltoBlue}\ifanswer \textbf{Solution.}#1\fi}}

%an environment to automatically hide answers. Wrap \InsertAnswer with this
\newenvironment{answer}{}

\newcommand{\createheader}[1]{
{\color{aaltoBlue}{
\LARGE\bf CS-E3190 Principles of Algorithmic Techniques \\ [2mm]
{\it #1 \ifgraded -- Graded Exercise \else -- Tutorial Exercise \fi}

}}}

\newcommand{\rulebox}[1]{
  \ifgraded

    \vspace{-3mm}
    {\color{aaltoBlue} \rule{16cm}{1pt}}

    \centering{
      \vspace{2mm}
      \fbox{\begin{minipage}{14.5cm}
          \vspace{2mm} \small \normalfont
          {Please read the following \textbf{rules} very carefully.
            \begin{itemize}[itemsep=0.5mm, leftmargin=15pt]
              \item Do not consciously search for the solution on the internet.
              \item You are allowed to discuss the problems with your classmates but you should \textbf{write the solutions yourself}.
              \item Be aware that \textbf{if plagiarism is suspected}, you could be asked to have an interview with teaching staff.
              \item The teaching staff can assist with understanding the problem statements, but will \textbf{not be giving any hints} on how to solve the exercises.
              \item The use of generative AI tools of any kind is \textbf{not allowed}.
              \item In order to ease grading, we want the solution of each problem and subproblem to start on a \textbf{new page}. If this requirement is not met, \textbf{points will be deduced}.

            \end{itemize}
          }
          \vspace{0.5mm}
        \end{minipage}}}
  \else
    \vspace{-3mm}
    {\color{aaltoBlue} \rule{16cm}{1pt}}
    \vspace{-2mm}
  \fi
}

%-----------------------------------------------
%-----------------------------------------------

\title{CS-E4850 Computer Vision D}
\author{Aleksi Kääriäinen  \\
	Aalto University  \\
	}

\begin{document}

\date{\today}

\maketitle

\newpage

\section*{Week 2 exercices}

\begin{enumerate}

    \item Solution in image

          \begin{figure}[H]
              \centering
              \includegraphics[scale=0.3]{../imgs/ex2_1.jpg}
          \end{figure}

          \newpage

    \item

          \begin{enumerate}

              \item To transform the image coordinates to pixel units, we need to scale the coordinates with the
                    pixel per unit distances and shift them with the principal point:
                    \begin{align*}
                        p = \begin{bmatrix}
                                m_u x_p + u_0 \\
                                m_v y_p + v_0
                            \end{bmatrix}
                    \end{align*}

              \item Since the $u$ axis is parallel to the $x$ axis, $p_u$ remains unchanged. The angle between
                    the $u$ and $v$ axis is $\theta$, so $m_v y_p$ needs to be scaled with the cosine of the angle.
                    \begin{align*}
                        p = \begin{bmatrix}
                                m_u x_p + u_0 \\
                                m_v y_p \cos{\theta} + v_0
                            \end{bmatrix}
                    \end{align*}

          \end{enumerate}
          \newpage

    \item Find a matrix $K_{3 \times 3}$, such that $\tilde{p} = K x_c$.
          \begin{align*}
              \begin{bmatrix}
                  m_u x_p + u_0 \\
                  m_v y_p + v_0 \\
                  1
              \end{bmatrix}                 & = K \begin{bmatrix}
                                                      x_c \\
                                                      y_c \\
                                                      z_c
                                                  \end{bmatrix}
              \intertext{Substitute equalities from exercice 1 for $x_p, y_p$}
              \begin{bmatrix}
                  m_u (f \frac{x_c}{z_c}) + u_0 \\
                  m_v (f \frac{y_c}{z_c}) + v_0 \\
                  1
              \end{bmatrix} & = K \begin{bmatrix}
                                      x_c \\
                                      y_c \\
                                      z_c
                                  \end{bmatrix}
              \intertext{Results in system of equations:}
                                               & \begin{cases*}
                                                     k_1 x_c + k_2 y_c + k_3 z_c = m_u (f \frac{x_c}{z_c}) + u_0 \\
                                                     k_4 x_c + k_5 y_c + k_6 z_c = m_v (f \frac{y_c}{z_c}) + v_0 \\
                                                     k_7 x_c + k_8 y_c + k_9 z_c = 1                             \\
                                                 \end{cases*} \\
                                               & \begin{cases*}
                                                     k_1 x_c + k_3 z_c = m_u (f \frac{x_c}{z_c}) + u_0 \\
                                                     k_5 y_c + k_6 z_c = m_v (f \frac{y_c}{z_c}) + v_0 \\
                                                     k_9 z_c = 1                                       \\
                                                 \end{cases*}           \\
              K                                & = \begin{bmatrix}
                                                       \frac{m_u f}{z_c} & 0                 & \frac{u_0}{z_c} \\
                                                       0                 & \frac{m_v f}{z_c} & \frac{v_0}{z_c} \\
                                                       0                 & 0                 & \frac{1}{z_c}
                                                   \end{bmatrix}   \\
              \intertext{Scale with $z_c$:}
              K                                & = \begin{bmatrix}
                                                       m_u f & 0     & u_0 \\
                                                       0     & m_v f & v_0 \\
                                                       0     & 0     & 1
                                                   \end{bmatrix}
          \end{align*}

          \newpage

    \item From lecture 1 slides:
          \begin{align*}
              P_{3 \times 4} & = K[R | t]                                                   \\
                             & = \begin{bmatrix}
                                     m_u f & 0     & u_0 \\
                                     0     & m_v f & v_0 \\
                                     0     & 0     & 1
                                 \end{bmatrix} \begin{bmatrix}
                                                   1 & 0 & 0 & 0 \\
                                                   0 & 1 & 0 & 0 \\
                                                   0 & 0 & 1 & 0 \\
                                               \end{bmatrix} \begin{bmatrix}
                                                                 r_{11} & r_{12} & r_{13} & t_1 \\
                                                                 r_{21} & r_{22} & r_{23} & t_2 \\
                                                                 r_{31} & r_{32} & r_{33} & t_3 \\
                                                                 0      & 0      & 0      & 1
                                                             \end{bmatrix} \\
              P              & = \begin{bmatrix}
                                     m_u f & 0     & u_0 \\
                                     0     & m_v f & v_0 \\
                                     0     & 0     & 1
                                 \end{bmatrix}\begin{bmatrix}
                                                  r_{11} & r_{12} & r_{13} & t_1 \\
                                                  r_{21} & r_{22} & r_{23} & t_2 \\
                                                  r_{31} & r_{32} & r_{33} & t_3 \\
                                              \end{bmatrix}
          \end{align*}

          \newpage

    \item

          \begin{enumerate}
              \item Any vector $x$ can be decomposed into two components:
                    \begin{itemize}
                        \item A parallel component wrt. rotation axis $u$: $x_{\parallel}$
                        \item A perpendicular component wrt. rotation axis $u$: $x_{\perp}$
                    \end{itemize}
                    \begin{align*}
                        x               & = x_{\parallel} + x_{\perp}
                        \intertext{The parallel component is the projection of $x$ onto $u$:}
                        x_{\parallel}   & = (u \cdot x) u
                        \intertext{And the perpendicular component is the difference of the vectors $x$ and $x_{\parallel}$}
                        x_{\perp}       & = x - x_{\parallel} = x - (u \cdot x) u
                        \intertext{The parallel component is unchanged after the rotation:}
                        R x_{\parallel} & = (u \cdot x) u
                        \intertext{The perpendicular component after rotation becomes:}
                        R x_{\perp}     & = \cos{\theta} x_{\perp} + \sin{\theta}(u \times x) = \cos{\theta} x - (u \cdot x) u + \sin{\theta}(u \times x)
                        \intertext{Combine the results:}
                        Rx              & = (u \cdot x) u + \cos{\theta} (x - (u \cdot x)u) + \sin{\theta}(u \times x)                                    \\
                        Rx              & = (u \cdot x) u + \cos{\theta}x - \cos{\theta} (u \cdot x)u + \sin{\theta}(u \times x)                          \\
                        Rx              & = \cos{\theta}x + \sin{\theta}(u \times x) + (1 - \cos{\theta})(u \cdot x)u
                    \end{align*}

              \item The cross product and dot product are:
                    \begin{align*}
                        u \times x   & = \begin{bmatrix}
                                             u_2 x_3 - u_3 x_2 \\
                                             u_3 x_1 - u_1 x_3 \\
                                             u_1 x_2 - u_2 x_1
                                         \end{bmatrix}   = \begin{bmatrix}
                                                               0    & - u_3 & u_2 \\
                                                               -u_3 & 0     & u_2 \\
                                                               -u_2 & u_1   & 0
                                                           \end{bmatrix} x = Kx                                                                   \\
                        (u \cdot x)u & = (u_1 x_1 + u_2 x_2 + u_3 x_3)u = \begin{bmatrix}
                                                                              u_1^2   & u_1 u_2 & u_1 u_3 \\
                                                                              u_1 u_2 & u_2^2   & u_2 u_3 \\
                                                                              u_1 u_3 & u_2 u_3 & u_3^2   \\
                                                                          \end{bmatrix} x = Ux
                        \intertext{Substitute these equalities into $Rx$:}
                        Rx           & = \cos{\theta}x + \sin{\theta}Kx + (1 - \cos{\theta})Ux
                        \intertext{$R$ is not dependent on $x$ anymore, divide $x$ out from both sides:}
                        R            & = \cos{\theta} I + \sin{\theta}K + (1 - \cos{\theta})U                                                     \\
                        R            & = \cos{\theta} \begin{bmatrix}
                                                          1 & 0 & 0 \\
                                                          0 & 1 & 0 \\
                                                          0 & 0 & 1 \\
                                                      \end{bmatrix} + \sin{\theta} \begin{bmatrix}
                                                                                       0    & - u_3 & u_2 \\
                                                                                       -u_3 & 0     & u_2 \\
                                                                                       -u_2 & u_1   & 0
                                                                                   \end{bmatrix} + (1 - \cos{\theta}) \begin{bmatrix}
                                                                                                                          u_1^2   & u_1 u_2 & u_1 u_3 \\
                                                                                                                          u_1 u_2 & u_2^2   & u_2 u_3 \\
                                                                                                                          u_1 u_3 & u_2 u_3 & u_3^2   \\
                                                                                                                      \end{bmatrix} \\
                    \end{align*}
          \end{enumerate}

          \newpage

\end{enumerate}
\end{document}
