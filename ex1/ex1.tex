\documentclass[11pt,a4paper]{article}
\input{../templates/auxiliary.tex}

%-----------------------------------------------
%-----------------------------------------------

\title{CS-E4850 Computer Vision D}
\author{Aleksi Kääriäinen  \\
	Aalto University  \\
	}

\begin{document}

\date{\today}

\maketitle

\newpage

\section*{Week 1 exercices}

\begin{enumerate}
      \item

            \begin{enumerate}

                  \item The equation of a line is $ax + by + c = 0$. The same equation can be represented in vector notation using $l = \begin{pmatrix}
                                    a & b & c
                              \end{pmatrix}^T$ and $x^T = \begin{pmatrix}
                                    x & y & 1
                              \end{pmatrix}$ and with the equation:
                        \begin{align*}
                              x^T l                         & = 0 \\
                              \begin{pmatrix}
                                    x & y & 1
                              \end{pmatrix} \begin{pmatrix}
                                                  a & b & c
                                            \end{pmatrix}^T & = 0 \\
                              ax + by + c                   & = 0
                        \end{align*}

                  \item By definition, the cross product $a \times b$ results in a vector that is perpendicular to
                        both of the input vectors. The result of the cross product can be interpreted as a homogenous coordinate point.
                        Thus the homogenous coordinate $x = l \times l^\prime$ lies on both lines $l$ and $l^\prime$, since it
                        fulfills both of the equations
                        \begin{align*}
                              x^T l          & = (l \times l^\prime)l        & = 0 \\
                              \text{and}                                           \\
                              x^T l^{\prime} & = (l \times l^\prime)l^\prime & = 0.
                        \end{align*}

                  \item The cross product $l=x \times x^\prime$ gives a vector $l$ that is perpendicular to the plane
                        defined by $x$ and $x^\prime$. This vector $l$ represents the line that passes through both points $x$ and $x^\prime$.
                        To check, we need to ensure that both points $x$ and $x^\prime$ lie on the line:
                        \begin{align*}
                              l^T x        & = (x \times x^\prime)^T x        & = 0, \\
                              \text{and}                                             \\
                              l^T x^\prime & = (x \times x^\prime)^T x^\prime & = 0.
                        \end{align*}

                        These equations hold because the cross product of two vectors is orthogonal to each of the original vectors.
                        Thus, the line through the points $x$ and $x^\prime$ is indeed given by $l = x \times x^\prime$.

                  \item From (c), it is known that the line through points $x$ and $x^\prime$ is $l = x \times x^\prime$.
                        To show that the point $y$ lies on the line, we need to show that:
                        \begin{align*}
                              l^T y                                                                     & = 0 \\
                              (x \times x^\prime)(\alpha x + (1 + \alpha)x^\prime)                      & = 0 \\
                              (x \times x^\prime)(\alpha x) + (x \times x^\prime)((1 + \alpha)x^\prime) & = 0 \\
                              \alpha(x \times x^\prime)x + (1 + \alpha)(x \times x^\prime)x^\prime      & = 0
                        \end{align*}
                        Since $x \times x^\prime$ is perpendicular to both $x$ and $x^\prime$, both terms in the equation
                        are 0 for any $\alpha \in \mathbb{R}$, meaning that $y$ is on the line.
            \end{enumerate}

            \newpage

      \item

            \newpage

      \item

            \newpage

\end{enumerate}
\end{document}
