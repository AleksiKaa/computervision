\documentclass[11pt,a4paper]{article}
\usepackage{hyperref}
\usepackage[left=1.5in,right=1.5in,top=1.5in,bottom=1.5in]{geometry}
\usepackage{enumitem}
\usepackage{amssymb,amsmath,mathrsfs,amsthm, empheq}
\usepackage{parskip, graphicx, float, wrapfig, subcaption}

\setlength{\parindent}{0pt}
\setlength{\parskip}{3mm}

\pagestyle{empty}

%Title and formatting stuff
\usepackage{titlesec}
\titleformat{\chapter}
{\normalfont\Huge\color{aaltoBlue}}{\thechapter}{20pt}{\filleft\Huge}[\vskip 0.5ex\titlerule]
\usepackage{fancyhdr}
\pagestyle{fancy}
\fancyhead[L]{}
\newcommand{\mymarginpar}[1]{\marginpar{\it \small \raggedleft #1}}
\renewcommand*\thesection{\arabic{section}}

% Miscellaneous LaTeX macros
\newcommand{\heading}[1]{{\Large\color{blue!80!black}\textbf{#1}}\par}
\newcommand{\slemp}[1]{{\color{red}#1}}
\newcommand{\slred}[1]{{\color{red}#1}}
\newcommand{\slbf}[1]{{\color{blue}\textbf{#1}}}
\newcommand{\pair}[2]{\langle {#1}, {#2} \rangle}
\newcommand{\Abar}{\bar{A}}
\newcommand{\ALG}{\text{ALG}}
\newcommand{\approxA}{\alpha_{\cal A}}
\newcommand{\argmin}{\operatornamewithlimits{arg\ min}}
\newcommand{\assign}{\mbox{$\; \leftarrow \;$}}
\newcommand{\bbar}{\bar{b}}
\newcommand{\Bin}{\text{Bin}}
\newcommand{\calA}{{\cal A}}
\newcommand{\calC}{{\cal C}}
\newcommand{\calI}{{\cal I}}
\newcommand{\calM}{{\cal M}}
\newcommand{\calN}{{\cal N}}
\newcommand{\calS}{{\cal S}}
\newcommand{\chat}{\hat{c}}
\newcommand{\chatopt}{\hat{c}^*}
\newcommand{\copt}{c^*}
\newcommand{\Cov}{\text{Cov}}
\newcommand{\DC}{\text{DC}}
\newcommand{\delC}{\partial C}
\newcommand{\delS}{\partial S}
\newcommand{\delT}{\partial T}
\newcommand{\diag}{\textit{diag}}
\newcommand{\dtv}{d_{\text{V}}}
\newcommand{\dom}{\mathcal{D}}
\newcommand{\E}{\text{E}}
\newcommand{\eps}{\varepsilon}
\newcommand{\false}{\text{false}}
\newcommand{\Gbar}{\bar{G}}
\newcommand{\Gnp}{{\cal G}(n,p)}
\newcommand{\Int}{\mathbf{I}}
\newcommand{\lambdamax}{\lambda_{\text{max}}}
\newcommand{\LPI}{$\text{LP}_{\text{I}}$}
\newcommand{\LPII}{$\text{LP}_{\text{II}}$}
\newcommand{\NN}{\mathbb{N}}
\newcommand{\om}{\omega}
\newcommand{\Om}{\Omega}
\newcommand{\OPT}{\text{OPT}}
\newcommand{\OPTf}{\text{OPT}_f}
\newcommand{\phase}{\bar{\sigma}}
\newcommand{\phat}{\hat{p}}
\newcommand{\pt}{p^{(t)}}
\newcommand{\rbot}{r^{\bot}}
\newcommand{\rhat}{\hat{r}}
\newcommand{\RENT}{\text{RENT}}
\newcommand{\RR}{\mathbb{R}}
\newcommand{\RRmn}{\mathbb{R}^{m\times n}}
\newcommand{\RRn}{\mathbb{R}^n}
\newcommand{\RRnn}{\mathbb{R}^{n\times n}}
\newcommand{\sgn}{\text{sgn}}
\newcommand{\stot}{$s$-$t$}
\newcommand{\true}{\text{true}}
\newcommand{\st}{\text{s.t.}}
\newcommand{\vivi}{v_i^Tv_i}
\newcommand{\vivj}{v_i^Tv_j}
\newcommand{\Var}{\text{Var}}
\newcommand{\xbar}{\bar{x}}
\newcommand{\xopt}{x^*}
\newcommand{\ybar}{\bar{y}}
\newcommand{\yopt}{y^*}
\newcommand{\zbar}{\bar{z}}
\newcommand{\zopt}{z^*}
\newcommand{\ZIP}{Z^*_{\text{IP}}}
\newcommand{\ZLP}{Z^*_{\text{LP}}}
\newcommand{\ZZ}{\mathbb{Z}}

\newcommand{\tD}[1]{\textrm{\em\color{aaltoGreen}{#1}}}
\newcommand{\tB}[1]{\text{\color{aaltoBlack}{#1}}}
\newcommand{\Pt}{\ensuremath{\mathrm{P}}}
\newcommand{\NP}{\ensuremath{\mathrm{NP}}}
\newcommand{\bra}{\ensuremath{\langle}}
\newcommand{\ket}{\ensuremath{\rangle}}
\newcommand{\DFT}{\ensuremath{\operatorname{DFT}}}
\newcommand{\rev}{\ensuremath{\operatorname{rev}}}
\newcommand{\quo}{\ensuremath{\operatorname{quo}}}
\newcommand{\rem}{\ensuremath{\operatorname{rem}}}
\newcommand{\take}{\ensuremath{\operatorname{\upharpoonright}}}
\newcommand{\ord}{\ensuremath{\operatorname{ord}}}
\newtheorem{fact}{Fact}
\newcommand\Edit{\mathrm{Edit}}
\newcommand\backtrack{\stackrel{\text{back-track}}{\longrightarrow}}

\newcommand\N{\mathbb{N}} % Sorry did not see \NN etc earlier and now my \N  is everywhere :(
\newcommand\R{\mathbb{R}}
\newcommand\Z{\mathbb{Z}}

%--------- definition environment commands --

\theoremstyle{definition}
\newtheorem{definition}{Definition}[section]
\newtheorem{lemma}{Lemma}[section]
\newtheorem{corollary}{Corollary}[section]
\newtheorem{theorem}{Theorem}[section]
\newtheorem{example}{Example}[section]

%--------- STUFF FOR 2021 ITERATION ---------

\usepackage{bbm}
\usepackage{float}
\usepackage{tikz}
\newcommand{\ex}{\text{E}}
\newcommand{\pr}{\text{P}}
\allowdisplaybreaks

% question title
\newcommand{\qtitle}[1]{{\color{aaltoBlue}\textbf{#1}}}

%% algorithm environment
\usepackage[ruled, vlined]{algorithm2e}
\newcommand\mycommfont[1]{\footnotesize\ttfamily\textcolor{aaltoBlue}{#1}}
\SetCommentSty{mycommfont}
\SetFuncSty{text}
\SetKwProg{Fn}{def}{:}{}
\usepackage{ulem}

\newif\ifanswer
\newif\ifgraded
\newcommand{\InsertAnswer}[1]{\vspace{1mm}{\color{aaltoBlue}\ifanswer \textbf{Solution.}#1\fi}}

%an environment to automatically hide answers. Wrap \InsertAnswer with this
\newenvironment{answer}{}

\newcommand{\createheader}[1]{
{\color{aaltoBlue}{
\LARGE\bf CS-E3190 Principles of Algorithmic Techniques \\ [2mm]
{\it #1 \ifgraded -- Graded Exercise \else -- Tutorial Exercise \fi}

}}}

\newcommand{\rulebox}[1]{
  \ifgraded

    \vspace{-3mm}
    {\color{aaltoBlue} \rule{16cm}{1pt}}

    \centering{
      \vspace{2mm}
      \fbox{\begin{minipage}{14.5cm}
          \vspace{2mm} \small \normalfont
          {Please read the following \textbf{rules} very carefully.
            \begin{itemize}[itemsep=0.5mm, leftmargin=15pt]
              \item Do not consciously search for the solution on the internet.
              \item You are allowed to discuss the problems with your classmates but you should \textbf{write the solutions yourself}.
              \item Be aware that \textbf{if plagiarism is suspected}, you could be asked to have an interview with teaching staff.
              \item The teaching staff can assist with understanding the problem statements, but will \textbf{not be giving any hints} on how to solve the exercises.
              \item The use of generative AI tools of any kind is \textbf{not allowed}.
              \item In order to ease grading, we want the solution of each problem and subproblem to start on a \textbf{new page}. If this requirement is not met, \textbf{points will be deduced}.

            \end{itemize}
          }
          \vspace{0.5mm}
        \end{minipage}}}
  \else
    \vspace{-3mm}
    {\color{aaltoBlue} \rule{16cm}{1pt}}
    \vspace{-2mm}
  \fi
}

%-----------------------------------------------
%-----------------------------------------------

\title{CS-E4850 Computer Vision D}
\author{Aleksi Kääriäinen  \\
	Aalto University  \\
	}

\begin{document}

\date{\today}

\maketitle

\newpage

\section*{Week 1 exercices}

\begin{enumerate}
      \item

            \begin{enumerate}

                  \item The equation of a line is $ax + by + c = 0$. The same equation can be represented in vector notation using $l = \begin{pmatrix}
                                    a & b & c
                              \end{pmatrix}^T$ and $x^T = \begin{pmatrix}
                                    x & y & 1
                              \end{pmatrix}$ and with the equation:
                        \begin{align*}
                              x^T l                         & = 0 \\
                              \begin{pmatrix}
                                    x & y & 1
                              \end{pmatrix} \begin{pmatrix}
                                                  a & b & c
                                            \end{pmatrix}^T & = 0 \\
                              ax + by + c                   & = 0
                        \end{align*}

                  \item By definition, the cross product $a \times b$ results in a vector that is perpendicular to
                        both of the input vectors. The result of the cross product can be interpreted as a homogenous coordinate point.
                        Thus the homogenous coordinate $x = l \times l^\prime$ lies on both lines $l$ and $l^\prime$, since it
                        fulfills both of the equations
                        \begin{align*}
                              x^T l          & = (l \times l^\prime)l        & = 0 \\
                              \text{and}                                           \\
                              x^T l^{\prime} & = (l \times l^\prime)l^\prime & = 0.
                        \end{align*}

                  \item The cross product $l=x \times x^\prime$ gives a vector $l$ that is perpendicular to the plane
                        defined by $x$ and $x^\prime$. This vector $l$ represents the line that passes through both points $x$ and $x^\prime$.
                        To check, we need to ensure that both points $x$ and $x^\prime$ lie on the line:
                        \begin{align*}
                              l^T x        & = (x \times x^\prime)^T x        & = 0, \\
                              \text{and}                                             \\
                              l^T x^\prime & = (x \times x^\prime)^T x^\prime & = 0.
                        \end{align*}

                        These equations hold because the cross product of two vectors is orthogonal to each of the original vectors.
                        Thus, the line through the points $x$ and $x^\prime$ is indeed given by $l = x \times x^\prime$.

                  \item From (c), it is known that the line through points $x$ and $x^\prime$ is $l = x \times x^\prime$.
                        To show that the point $y$ lies on the line, we need to show that:
                        \begin{align*}
                              l^T y                                                                     & = 0 \\
                              (x \times x^\prime)(\alpha x + (1 + \alpha)x^\prime)                      & = 0 \\
                              (x \times x^\prime)(\alpha x) + (x \times x^\prime)((1 + \alpha)x^\prime) & = 0 \\
                              \alpha(x \times x^\prime)x + (1 + \alpha)(x \times x^\prime)x^\prime      & = 0
                        \end{align*}
                        Since $x \times x^\prime$ is perpendicular to both $x$ and $x^\prime$, both terms in the equation
                        are 0 for any $\alpha \in \mathbb{R}$, meaning that $y$ is on the line.
            \end{enumerate}

            \newpage

      \item

            \newpage

      \item

            \newpage

\end{enumerate}
\end{document}
